\documentclass[conference]{IEEEtran}
\IEEEoverridecommandlockouts
% The preceding line is only needed to identify funding in the first footnote. If that is unneeded, please comment it out.
\usepackage{cite}
\usepackage{amsmath,amssymb,amsfonts}
\usepackage{algorithmic}
\usepackage{graphicx}
\usepackage{textcomp}
\usepackage{xcolor}
\def\BibTeX{{\rm B\kern-.05em{\sc i\kern-.025em b}\kern-.08em
    T\kern-.1667em\lower.7ex\hbox{E}\kern-.125emX}}
\begin{document}

\title{U-Net: Uma rede convolucional voltada para segmentação de imagens biomédicas}

\author{\IEEEauthorblockN{1\textsuperscript{st} Marcus Vinícius de Faria Santos}
\IEEEauthorblockA{\textit{Centro de Informática} \\
\textit{Universidade Federal de Pernambuco}\\
Recife, Brasil \\
mvfs@cin.ufpe.br}
\and
\IEEEauthorblockN{2\textsuperscript{nd} Lucas Nascimento Brandão}
\IEEEauthorblockA{\textit{Centro de Informática} \\
\textit{Universidade Federal de Pernambuco}\\
Recife, Brasil \\
bmmuc@cin.ufpe.br}
\and
\IEEEauthorblockN{2\textsuperscript{rd} Jefferson Severino de Araújo}
\IEEEauthorblockA{\textit{Centro de Informática} \\
\textit{Universidade Federal de Pernambuco}\\
Recife, Brasil \\
bmmuc@cin.ufpe.br}
\and
\IEEEauthorblockN{2\textsuperscript{th} Rodrigo Rocha Moura}
\IEEEauthorblockA{\textit{Centro de Informática} \\
\textit{Universidade Federal de Pernambuco}\\
Recife, Brasil \\
bmmuc@cin.ufpe.br}
\and
\IEEEauthorblockN{2\textsuperscript{th} Rodrigo Rocha Moura}
\IEEEauthorblockA{\textit{Centro de Informática} \\
\textit{Universidade Federal de Pernambuco}\\
Recife, Brasil \\
bmmuc@cin.ufpe.br}
}

\maketitle

\begin{abstract}
Este documento tem por objetivo a explicação, de modo suscinto, sobre a rede convolcuional U-Net para segmentação de imagens. Ele foi desenvolvido para a disciplina IF867- Introdução à Aprendizagem Profunda (2022.2) e será dividido em 'Introdução', 'História', 'Como Funciona?', 'Aplicações e Exemplos' e 'Referências'. 
\end{abstract}

\begin{IEEEkeywords}
Rede Convolucional, Segmentação de imagens, U-Net
\end{IEEEkeywords}

\section{Introdução}
No momento da criação da rede U-Net, era um consenso que para que fossem bem-sucedidas, as redes profundas deveriam ser treinadas com milhares de imagens catalogadas. Contudo, isso era um grande problema, pois nem todos os problemas que demandavam esse tipo de solução tinham grande disponibilidade de dados. Objetivando superar esse problema, os criadores dessa abordagem propuseram um novo modelo em que se utilizavam de Data Augmentation de forma intensa nas poucas imagens que tinham para aumentar a quantidade de imagens para treino. O resultado obtido foi uma nova rede, que devido ao seu formato apelidaram de U-Net, que foi vencedora de duas competições ISBI.

\begin{figure}[h]
    \centering
    \includegraphics[width=\linewidth]{U-net.png}
    \caption{Arquitetura U-Net, exemplo para um modelo com 32x32 pixels na menor resolução}
    \label{fig:U_Net}
\end{figure}


\section{História}


\section{Arquitetura}

\section{Aplicações e Exemplos}

\subsection{Problema}

\paragraph{YOLO}




\begin{thebibliography}{00}
\bibitem{b1} 
\end{thebibliography}
\vspace{12pt}

\end{document}
